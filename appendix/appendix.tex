\chapter{Ideal particle hysteresis code}
\label{ch:appendix}
\fancyhead[L]{Appendix A. Ideal particle hysteresis code}
\fancyhead[C]{}
\fancyhead[R]{}
\fancyfoot[C]{\thepage}

\lstset{language=Python}
\definecolor{dkgreen}{rgb}{0,0.6,0}
\definecolor{gray}{rgb}{0.5,0.5,0.5}
\definecolor{mauve}{rgb}{0.58,0,0.82}
\lstset{frame=tb,
  language=Python,
  aboveskip=3mm,
  belowskip=3mm,
  showstringspaces=false,
  columns=flexible,
  basicstyle={\small\ttfamily},
  numbers=none,
  numberstyle=\tiny\color{gray},
  keywordstyle=\color{blue},
  commentstyle=\color{dkgreen},
  stringstyle=\color{mauve},
  breaklines=true,
  breakatwhitespace=true,
  tabsize=3
}

For Chapter \ref{ch:res-2}, a conjugate gradient numerical method with line search was developed in the language Python. A minimal working example follows. This code can be run as:
\begin{verbatim}
python -i SWCAFORC.py -m greigite --applied_field 0.247 0.474 0.845 -f 1
\end{verbatim}
for an interactive Python session that will produce two plots: (1) the hysteresis loop and first-order reversal curves (FORCs) and (2) a plot of the FORC distribution for an idealised greigite particle with applied field in the direction of the vector $(0.247\ihat + 0.474\jhat + 0.845\boldsymbol{\hat{k}})$ and a smoothing factor SF=1.

\begin{lstlisting}
#!/usr/bin/env python

from argparse import ArgumentParser
from numpy import sin, cos, power, sqrt, dot, meshgrid, min, max
from numpy import array, linspace, arange, arccos, pi, c_, hstack
from numpy import zeros, zeros_like, ones, prod, concatenate
from random import uniform
from scipy.linalg import lstsq
import matplotlib.pyplot as plt
from matplotlib.cm import RdBu_r as cmap
from matplotlib.colors import LinearSegmentedColormap

def main():
  # Parse args.
  ap = ArgumentParser(
      description=(
      'A simple Stoner-Wohlfarth model with cubic anisotropy. '         +\
      'Perform a hysteresis loop with saturation field strength 250 mT.'+\
      'Followed by the necessary FORCs.')
      )
  ap.add_argument('-m', '--material',
                  type= str,
                  choices= ['greigite', 'magnetite', 'iron'],
                  default= 'greigite',
                  help= 'The material. Greigite, magnetite or iron (defaults to greigite).',
                  )
  ap.add_argument('-a', '--applied_field',
                  nargs= 3,
                  type= float,
                  help= 'A vector in the direction of the field.',
                  )
  ap.add_argument('-f', '--factor',
                   type= int,
                   default= 1,
                   help= 'The smoothing factor (defaults to 1).'
                  )

  args = ap.parse_args()
  material = args.material
  field = args.applied_field
  SF = args.factor

  #Magnetic parameters.
  params = {'greigite' : {'K1': -1.7e4, 'Ms': 2.702e5, 'K2': 0.},
            'magnetite': {'K1': -1.2e4, 'Ms': 4.8e5   , 'K2': 0.},
            'iron'     : {'K1':  4.8e4, 'Ms': 1.715e6 , 'K2': 0.},
            }
  K1 = params[material]['K1']
  Ms = params[material]['Ms']
  K2 = params[material]['K2']

  # Normalise the field direction.
  field = array(field)
  field = field/sqrt(dot(field, field))

  # Applied Field. In Tesla.
  start =  0.25
  end   = -1.*start
  applied = concatenate((linspace(start,     end,   501),
                            linspace(end+0.001, start, 500))
  )

  # Initial guess. A random point from a uniform distribution over the sphere.
  u     = uniform(0., 1.)
  v     = uniform(0., 1.)
  phi   = 2.*pi*u
  theta = arccos(2.*v - 1.)

  # The gradient at the initial guess.
  g = grad(theta,
           phi,
           field,
           B=start,
           K1=K1,
           K2=K2,
           Ms=Ms
  )

  # The energy of the initial guess.
  nrg = energy(theta,
               phi,
               field,
               B=start,
               K1=K1,
               K2=K2,
               Ms=Ms
  )

  # Energy tolerance.
  etol = 1e-12

  # Gradient tolerance.
  gtol = 1e-12

  # Armijo-Goldstein parameters.
  tau = 1./2.
  c = 1e-4

  mvectors = zeros((1001,3))
  path     = zeros((1001,2))
  mag      = zeros(1001)

  # Main branch hysteresis:
  print 'Calculating hysteresis loop'
  for i, B in enumerate(applied):
    theta, phi, nrg = minimise(theta,
                               phi,
                               field,
                               B=B,
                               K1=K1,
                               K2=K2,
                               Ms=Ms,
                               tau=tau,
                               c=c,
                               gtol=gtol,
                               etol=etol,
    )
    m             = to_cartesian(theta, phi)
    mvectors[i]   = m
    m_along_field = dot(m, field)
    mag[i]        = m_along_field
    print 'Field=%5.1f mT, m=%9.6f' % (B*1e3, m_along_field)

  # Find the discontinuities to calculate the necessary FORCs:
  discontinuities = [array((applied[i], i)) for i in range(500) if angle(mvectors[i], mvectors[i+1]) > 5.*pi/180.]
  discontinuities = array(discontinuities)

  mFORCs = []
  BFORCs = []
  for fieldvalue, index in discontinuities[1:]:
    print 'Calculating FORC starting at Ba=%5.1f mT' % (fieldvalue*1e3)
    magF = []
    BF   = []
    theta_old, phi_old = path[int(index)]
    for B in arange(fieldvalue, start+0.001, 0.001):
      theta, phi, nrg = minimise(theta,
                                 phi,
                                 field,
                                 B=B,
                                 K1=K1,
                                 K2=K2,
                                 Ms=Ms,
                                 tau=tau,
                                 c=c,
                                 gtol=gtol,
                                 etol=etol,
      )
      m_along_field = dot(to_cartesian(theta, phi), field)
      magF.append(m_along_field)
      BF.append(B)
      print 'Ba=%5.1f, Bb=%5.1f mT, m=%9.6f' % (fieldvalue*1e3, B*1e3, m_along_field)

    mFORCs.append(array(magF))
    BFORCs.append(array(BF))

  mFORCs = array(mFORCs)
  BFORCs = array(BFORCs)

  # Plot the hysteresis loop and FORCs
  fig, ax = plt.subplots()
  plt.plot(applied*1e3, mag)
  for mforc, bforc in zip(mFORCs, BFORCs):
    plt.plot(bforc*1e3, mforc)

  plt.xlabel(r'$B\,(\mathrm{mT})$')
  plt.ylabel(r'$m$')
  plt.show(block=False)

  # Calculate the FORC distribution.
  # First, put all the FORC and hysteresis data into one 1D array.
  mFORC = zeros(sum(range(501+1)))

  # First part, using the first discontinuity.
  for i in range(int(discontinuities[0][1]+1)):
    mFORC[FORCcounter(i)] += mag[i]
    for j in range(i):
      mFORC[FORCcounter(i)+j+1] += mag[i-j-1]

  # Using the rest of the discontinuities.
  for i in range(len(discontinuities[1:])):
    for j in range(int(discontinuities[i][1])+1, int(discontinuities[i+1][1])+1):
      mFORC[FORCcounter(j)] += mFORCs[i][int(discontinuities[i+1][1])-j]
      for k in range(j):
        mFORC[FORCcounter(j)+k+1] += mFORCs[i][int(discontinuities[i+1][1]-j)+k+1]

  # Using the lower main branch.
  for i in range(int(discontinuities[-1][1])+1, 501):
    mFORC[FORCcounter(i)] += mag[500:][501-i-1]
    for j in range(i):
      mFORC[FORCcounter(i)+j+1] += mag[500:][501-i+j]

  # Rearrange the hysteresis and FORC data onto a 2D array.
  Bb, Ba = meshgrid(linspace(-250., 250., 501), linspace(250., -250., 501))
  m = zeros_like(Bb)
  for i in xrange(501):
    for j, k in enumerate(xrange(sum(range(i)), sum(range(i+1)))):
      m[i][500-i+j] = mFORC[sum(range(i))+j]

  # Calculate the forc-distribution.
  rho = zeros_like(m)
  for i in xrange(501):
    for j in xrange(500-i, 501):
      grid = in_grid(i, j, SF)
      data = []
      for indices in grid:
        points = (Bb[indices[0]][indices[1]],
                  Ba[indices[0]][indices[1]],
                  m[indices[0]][indices[1]],
                  )
        points = array(points)
        data.append(points)
      data = array(data)
      A = c_[ones(data.shape[0]), data[:,:2], prod(data[:,:2], axis=1), data[:,:2]**2]
      C,_,_,_ = lstsq(A, data[:,2])
      rho[i][j] += -C[3]/2.

  # Normalise the forc distribution.
  rhomax = max(rho)
  rho /= rhomax
  rhomin = min(rho)
  rhomax = max(rho)
  shiftedCMap = shiftedColorMap(cmap, midpoint=(1. - rhomax/(rhomax + abs(rhomin))))

  # A simple plot, no contours.
  fig, ax = plt.subplots()
  plt.pcolor(Bb, Ba, rho, cmap=shiftedCMap)
  plt.plot([   0., 250.], [   0., -250.], color= 'black', linestyle='--')
  plt.plot([-250., 250.], [-250.,  250.], color= 'black')
  clb = plt.colorbar()
  clb.ax.set_title(r'$\rho^{*}$')
  plt.xlabel(r'$B_b\,(\mathrm{mT})$')
  plt.ylabel(r'$B_a\,(\mathrm{mT})$')
  plt.show(block=False)


def FORCcounter(x):
  if x == 0:
    return 0
  else:
    return x + FORCcounter(x-1)


def in_grid(i, j, sf=1):
  grid = []
  for k in range(i-sf, i+sf+1):
    for l in range(j-sf, j+sf+1):
      if in_triangle(k, l) and in_square(k, l):
        grid.append((k, l))
  return tuple(grid)


def in_triangle(i, j):
  return (True if j >= (500-i) else False)


def in_square(i, j):
  return (True if (i<=500 and j<=500) else False)


def energy(theta,
           phi,
           field,
           B=0.,
           K1=-1.7e4,
           K2= 0.,
           Ms=2.2706e5
           ):
  xi, psi, omega = field
  return ((K1/Ms)*power(sin(theta), 2)*(
          power(cos(theta), 2)+power(sin(theta)*cos(phi)*sin(phi), 2)) +
          (K2/Ms)*power(power(sin(theta), 2)*cos(theta)*sin(phi)*cos(phi), 2) -
          B*(xi*sin(theta)*cos(phi)+psi*sin(theta)*sin(phi)+omega*cos(theta)
          )
  )


def grad(theta,
         phi,
         field,
         B=0.,
         K1=-1.7e4,
         K2= 0.,
         Ms=2.2706e5
         ):
  xi, psi, omega = field
  ehat_theta = (2.*(K1/Ms)*sin(theta)*cos(theta)*(
                2.*power(sin(theta)*sin(phi)*cos(phi), 2) -
                power(sin(theta), 2)+power(cos(theta), 2)) +
                2.*(K2/Ms)*sin(theta)*cos(theta)*(
                power(sin(phi)*cos(phi), 2)*(2.*power(sin(theta)*cos(theta), 2)- power(sin(theta), 4))) -
                B*(xi*cos(theta)*cos(phi)+psi*cos(theta)*sin(phi)-omega*sin(theta))
                )
  ehat_phi = (2.*(K1/Ms)*power(sin(theta), 4)*sin(phi)*cos(phi)*(
              power(cos(phi), 2)-power(sin(phi), 2)) +
              2.*(K2/Ms)*power(sin(theta), 4)*sin(phi)*cos(phi)*(
              power(cos(theta), 2)*(power(cos(phi), 2)-power(sin(phi), 2))) -
              B*sin(theta)*(psi*cos(phi)-xi*sin(phi))
              )
  return array((ehat_theta, ehat_phi))


def linesearch(f,
               g,
               theta,
               phi,
               field,
               B=0.,
               K1=-1.7e4,
               K2=0.,
               Ms=2.2706e5,
               tau=0.5,
               c=1e-4,
               ):
  xi, psi, omega = field
  gamma = 1.
  while f(theta-gamma*g[0], phi-gamma*g[1], field, B=B, K1=K1, K2=K2, Ms=Ms) > f(theta, phi, field, B=B, K1=K1, K2=K2, Ms=Ms) - c*gamma*dot(g, g):
    gamma *= tau
  return gamma


def minimise(theta,
             phi,
             field,
             B=0.,
             K1=-1.7e4,
             K2=0.,
             Ms=2.2706e5,
             tau=0.5,
             c=1e-4,
             gtol=1e-12,
             etol=1e-12,
             ):
  g = grad(theta, phi, field, B=B, K1=K1, K2=K2, Ms=Ms)
  nrg = energy(theta, phi, field, B=B, K1=K1, K2=K2, Ms=Ms)
  
  nrg_old = 1e10  
  while (dot(g, g) > gtol or Ms*abs(nrg_old-nrg) > etol):
    gamma = linesearch(energy,
                       g,
                       theta,
                       phi,
                       field,
                       B=B,
                       K1=K1,
                       K2=K2,
                       Ms=Ms,
                       c=c,
                       tau=tau,
    )
    theta_old, phi_old = theta, phi
    theta += -gamma*g[0]
    phi += -gamma*g[1]

    nrg_old = energy(theta_old,
                     phi_old,
                     field,
                     B=B,
                     K1=K1,
                     K2=K2,
                     Ms=Ms
    )
    nrg = energy(theta,
                 phi,
                 field,
                 B=B,
                 K1=K1,
                 K2=K2,
                 Ms=Ms
    )
    g = grad(theta,
             phi,
             field,
             B=B,
             K1=K1,
             K2=K2,
             Ms=Ms
    )
  return (theta, phi, nrg)


def to_cartesian(theta, phi):
  return array((sin(theta)*cos(phi), sin(theta)*sin(phi), cos(theta)))


def angle(u, v):
    return arccos(dot(u, v))


def shiftedColorMap(cmap, start=0, midpoint=0.5, stop=1.0, name='shiftedcmap'):
    '''
    Function to offset the "center" of a colormap. Useful for
    data with a negative min and positive max and you want the
    middle of the colormap's dynamic range to be at zero

    Input
    -----
      cmap : The matplotlib colormap to be altered
      start : Offset from lowest point in the colormap's range.
          Defaults to 0.0 (no lower ofset). Should be between
          0.0 and `midpoint`.
      midpoint : The new center of the colormap. Defaults to 
          0.5 (no shift). Should be between 0.0 and 1.0. In
          general, this should be  1 - vmax/(vmax + abs(vmin))
          For example if your data range from -15.0 to +5.0 and
          you want the center of the colormap at 0.0, `midpoint`
          should be set to  1 - 5/(5 + 15)) or 0.75
      stop : Offset from highets point in the colormap's range.
          Defaults to 1.0 (no upper ofset). Should be between
          `midpoint` and 1.0.
    '''
    cdict = {
        'red': [],
        'green': [],
        'blue': [],
        'alpha': []
    }

    # regular index to compute the colors
    reg_index = linspace(start, stop, 257)

    # shifted index to match the data
    shift_index = hstack([
        linspace(0.0, midpoint, 128, endpoint=False), 
        linspace(midpoint, 1.0, 129, endpoint=True)
    ])

    for ri, si in zip(reg_index, shift_index):
        r, g, b, a = cmap(ri)

        cdict['red'].append((si, r, r))
        cdict['green'].append((si, g, g))
        cdict['blue'].append((si, b, b))
        cdict['alpha'].append((si, a, a))

    newcmap = LinearSegmentedColormap(name, cdict)
    plt.register_cmap(cmap=newcmap)

    return newcmap


if __name__ == '__main__':
  main()
\end{lstlisting}
