\cleardoublepage
\phantomsection
\addcontentsline{toc}{chapter}{Abstract}

\begin{abstract}
The iron sulphide greigite ($\text{Fe}_3\text{S}_4$) is linked to important processes in sediments and to hydrocarbon formation and migration. This ferrimagnetic mineral is relatively obscure because of difficulties to produce in laboratories and because it is unstable and therefore thought to be irrelevant to the geological record. However, it is increasingly recognised that greigite can remain stable on geological timescales. It is important to understand the magnetic properties of greigite to identify its presence and timing of formation as it is a proxy for environmental magnetic studies.\par
In this thesis, numerical methods are used to study the magnetic properties of greigite. Using a micromagnetic finite-element method (FEM), important questions regarding the magnetic structure and palaeomagnetic recording fidelity of gregite are addressed. For equidimensional particles, the single-domain (SD) to single-vortex (SV) threshold is found to be $d_0\approx54\nm$ and only SV particles $>70\nm$ to carry stable magnetisations over billion-year timescales. A simplified model is developed to study the hysteresis and first-order reversal curve (FORC) properties of non-interacting idealised SD greigite particles. To understand the effects of SV magnetisations on FORC properties, a micromagnetic FEM is used to simulate randomly oriented dispersions of non-interacting greigite in the SV size range. SV effects dominate the FORC signal for particles $>70\nm$. Implications for FORC diagram interpretation are discussed. Magnetic inter-particle interactions are known to effect the FORC response of magnetic particle ensembles. A micromagnetic FEM is used to study the FORC signal of randomly dispersed strongly interacting clusters of greigite. The FORC response of strongly interacting greigite is found to be similar to that of multi-domain (MD) particles. Since naturally occurring greigite is rarely in a MD state, it is concluded that in greigite-bearing rocks that produce MD-like FORC signals, this signal should be attributed to strong interactions between the particles.
\end{abstract}
