\addcontentsline{toc}{chapter}{Abstract}

\begin{abstract}
Magnetic iron oxide and iron sulphide minerals are responsible for the magnetisation of certain types of rocks. Natural remanent magnetisation (NRM) of rocks can provide an invaluable wealth of knowledge to Earth scientists, e.g., volcanic and sedimentary rocks are known to carry a record of Earth's magnetic field's past (palaeomagnetism). Also, small variations in magnetic particle abundance and size distributions in sediments and sedimentary rocks are related to palaeoclimatic conditions (environmental magnetism). The bulk magnetic properties of rocks are greatly influenced by the size of the magnetic particles; small (nanometric) particles carry strong, uniform magnetisations in a single-domain (SD) state, whereas large (micronic) particles have low magnetisations due to their non-uniform multi-domain (MD) structure. Transition from SD to MD behaviour is smooth, occurring through the so-called pseudo single-domain (PSD) particle size range, spanning an order of magnitude. These have magnetic properties between SD and MD particles and are the magnetic carriers in many systems. In the lower end of the PSD range, PSD particles are single-vortex (SV) magnetisations for which there is not a comprehensive theory that reflects the physics of this type of magnetic domain state.\par
The iron sulphide greigite (Fe$_3$S$_4$) has been linked to important processes in sediments and to hydrocarbon formation and migration. This mineral is relatively obscure because of difficulties to produce in laboratories and because it has been thought to be untable and therefore irrelevant to the geological record. However, it is increasingly recognised that greigite can remain stable on geological scales. It is important to understand the magnetic properties of greigite and how to identify its presence and timing of formation as it can either carry a reliable palaeomagnetic signal or it can remagnetise the host rock.\par
In this thesis, numerical methods are used to study the magnetic properties of the iron sulphide greigite. Using a micromagnetic finite element method (FEM) important questions regarding the magnetic structure and palaeomagnetic recording fidelity of gregite are addressed. The SD--SV threshold is found to be $\roughly$54$\nm$ and only SV particles $>$70$\nm$ to carry stable magnetisations on a geological timescale. A simplified model is developed to study the hysteresis and first-order reversal curve (FORC) properties of non-interacting idealised SD greigite particles. Unique FORC properties of SD particles with cubic MCA are proposed. To understand the effects of SV magnetisations on FORC properties, a micromagnetic FEM is used to simulate randomly oriented dispersions of non-interacting greigite in the SV range. SV effects are found to dominate the FORC signal for particles $>$70$\nm$ and implications for the interpretation of FORC diagrams for SV particles are discussed. Magnetic inter-particle interactions are known to effect the FORC response of magnetic particle ensembles. A micromagnetic FEM is used to study the FORC signal of randomly dispersed strongly-interacting clusters of greigite with individual particles in the SD range. The FORC signal of strongly-interacting greigite is found to be MD-like. Since naturally occurring greigite is rarely as large as to be MD, it is concluded that in rocks known to contain greigite that produce MD-like FORC signals, this signal should be attributed to strong interactions between the particles.
\end{abstract}
